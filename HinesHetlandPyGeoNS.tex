\documentclass[10pt,a4paper]{article}
\usepackage[utf8]{inputenc}
\usepackage{amsmath}
\usepackage{amsfonts}
\usepackage{amssymb}
\usepackage{natbib}
\usepackage{graphicx}
\usepackage[left=2cm,right=2cm,top=2cm,bottom=2cm]{geometry}
\usepackage{arydshln}
\begin{document}

\section{Introduction}\label{sec:Introduction}

Crustal strain rates are fundementally important quantities for assessing seismic hazard. This is because the locations where strain is rapidly accumulating are the locations where we can expect strain energy to be released seismically.  It is then important to develop and improve upon methods for mapping strain in tectonically active regions because such maps could concievably feed into seismic hazard models such as UCERF3 \citep{Field2014}. 

Maps of strain rate can be derived from geodetic measurements of ground displacements, and there are numerous methods for doing so.  The classic and simplest way is to assume that the strain rate is constant in time and spatially uniform within subnetworks of the geodetic data.  Linear least squares is then used to find the components of the strain rate tensor for each subnetwork \citep[e.g][]{Frank1966,Prescott1976,Savage1986,Feigl1993,Murray2000}. Several algorithms have been developed to improve upon this procedure for calculating strain. \citet{Shen1996} and \citet{Shen2015} discuss an algorithm where, instead of using the immediately adjacent stations to calculate strain at a position, the strain is computed with a weighted average over the entire network where the weighting is smaller for more distant stations.  Another strategy is to fit a set of interpolating basis functions to the velocity field and then compute the strain from the analytical derivative of the interpolant \citep[e.g.][]{Beavan2001,Tape2009}.  The aforementioned studies have all been concerned with estimating long term strain rates.  Time dependent strain would be useful for studying geophysical processes which occur over timescales of days to years such as slow slip events, postseismic relaxation, or volcanic deformation.  \citet{Ohtani2010} describes a Kalman filter based method for computing time dependent strain by fitting a set of basis functions to a time dependent displacement field and enforcing temporal smoothness in the basis function coefficients.   

In essence, estimating strain rates is a matter of numerically calculating the spatial derivative of a geodetically observed velocity field.  Any method proposed for calculating strain rates must be able to handle two complications; 1) geodetic velocity estimates are noisy and differentiation will only amplify the noise and 2) velocities are not observed on a regular grid, which prevents the use of standard finite difference methods for computing derivatives.  In this paper we demonstrate that both of these complications can be elegantly handled with the recently popularized Radial Basis Function-Finite Difference (RBF-FD) method \citep{Wright2006}.  

The RBF-FD method was introduced as an computationally efficient way to solve large scale partial differential equations over irregular, multi-dimensional domains.  The RBF-FD method can be thought of as a generalization of the traditional finite difference method, where the node layout is no longer restricted to regular grids. Indeed, the RBF-FD method can be used to estimate derivates of discrete data located at arbitrary scattered positions in multi-dimensional space.  The RBF-FD method is particularly appealing because it is algorithmically simple, regardless of the domain shape or node layout, and also because the method has performed well in numerous benchmark tests \citep[and references therein]{Fornberg2015}.

In this paper, we do not use the RBF-FD method to solve a partial differential equation, but rather we use it to spatially smooth and differentiate GPS derived velocity data.  Our smoothing strategy can be viewed  as a low-pass filter for scattered data where the degree of smoothness is controlled by a user specifies cutoff frequency.  This can be contrasted with interpolation based smoothing strategies \citep[e.g.][]{Howell2016} where the resulting interpolant can be largely and unpredictably controlled by the choice of basis function.  After spatially smoothing the velocity field we differentiate it with the RBF-FD method to get a strain rate map.  We also demonstrate that this procedure can be used to estimate time dependent strain rates.  In that case, we first temporally smooth and differentiate GPS displacement time series to get time dependent velocities.  We then spatially smooth and differentiate the resulting velocities for each time epoch.  

The method proposed in this paper has numerous advantages which set it apart from other methods for computing strain rates. The method is computationally efficient and stable (there is no inversion of an ill-conditioned matrix).  There are no hyper parameters or penalty parameters that need to be tuned for each application.  As opposed to interpolation strategies such as \citet{Beavan2001}, \citet{Tape2009}, or \citet{Ohtani2010}, our method assumes that velocities are locally rather than globally continuous, which allows us to easily handle discontinuities resulting from, for example, a creeping fault. 

We begin this paper by summarizing the RBF-FD method and explaining how we construct differentiation matrices for scattered data. We then introduce the smoothing strategy, which is applied to the observed geodetic data prior to differentiation.  We then provide two real world demonstraitions of our method for calculating strain rates.  First we calculate the long term strain rates in Southern California from the CMM3 velocity data set \citep{Shen2011}, and we verify that our results are consistent with other studies. We then calculate time dependent strain rates in Cascadia from the GPS data provided by UNAVCO.  In Cascadia, we analyze strain resulting from slow slip events and compare it to the long term tectonic strain accumulation. Slow slip events are found to produce compression in the Olympic Peninsula, which is in addition to the compression resulting from tectonic loading.  Further south in Oregon, the slow slip events tend to release the compressional strain that is accumulated tectonically.  While similar conclusions have been drawn from fault slip inversions for slow slip events, it is important to recognize that slip inversion are the product of inverting an ill-conditioned matrix making it difficult to determine whether slip inferences are real or just an artifact of the inversion.  The strain rates presented in this paper are more direct observations and can be interpretted with a higher degree of confidence. 

\section{Method}\label{sec:Method}
\subsection{Differentiating Scattered Data}\label{sec:Differentiating}

In this section we briefly summarizing the RBF-FD method and we refer the reader to \citet{Wright2006} or \citet{Fornberg2015} for additional details. Consider a set of nodes $\mathbf{x} = \{x_1, ..., x_N\}$ in $\mathbb{R}^d$ and corresponding observations of a scalar valued functions $\mathbf{u} = \{u(x_1), ..., u(x_N)\}$.  We want to find a differentiation matrix $\mathbf{L}$, such that $\mathbf{Lu}$ approximates the linear differential operator $\mathcal{L}$ acting on $u$ at nodes $x_1, \dots, x_N$.  For each node $x_i$ we approximate $\mathcal{L}[u(x)]\big|_{x=x_i}$ as a weighted sum of $\{u(x_j): j \in \mathcal{S}_i\}$ where $\mathcal{S}_i$ consist of $i$ and the subscripts for the $n-1$ nearest neighboring nodes to $x_i$.  The approximation can be written as

\begin{equation}\label{eq:RBFFDApprox}
\mathcal{L}[u(x)]\big|_{x=x_i} \approx \sum_{j \in \mathcal{S}_i} L_{ij} u(x_j)
\end{equation}
where $L_{ij}$ are the weights making up the differentiation matrix $\mathbf{L}$.  We refer to $x_i$ and its $n-1$ nearest neighbors as the stencil for $x_i$, and it is denoted as $\mathbf{x}^i = \{x_j : j \in \mathcal{S}_i\} = \{x^i_1,..., x^i_n\}$. The corresponding weights for each node in $\mathbf{x}^i$ are denoted as $\mathbf{w}^i = \{L_{ij} : j \in \mathcal{S}_i\} = \{w^i_1,...w^i_n\}$.  We can then equivalently write eq. (\ref{eq:RBFFDApprox}) as 

\begin{equation}\label{eq:RBFFDApproxAlt}
\mathcal{L}[u(x)]\big|_{x=x_i} \approx \sum_{j=1}^n w^i_j u(x^i_j).
\end{equation}
Following \citet{Fornberg2015}, we find the components of $\mathbf{w}^i$, and thus also of $\mathbf{L}$, by solving the linear system of equations

\begin{equation}\label{eq:RBFFDWeights}
\left[
      \begin{array}{ccc:ccc}
      \phi(||x^i_1 - x^i_1||) & \cdots & \phi(||x^i_n - x^i_1||) & \psi_1(x^i_1) & \cdots & \psi_m(x^i_1) \\
      \vdots                  &        & \vdots                  & \vdots        &        & \vdots        \\   
      \phi(||x^i_1 - x^i_n||) & \cdots & \phi(||x^i_n - x^i_n||) & \psi_1(x^i_n) & \cdots & \psi_m(x^i_n) \\  
                              &        &                         &               &        &               \\ \hdashline 
                              &        &                         &               &        &               \\
      \psi_1(x^i_1)           & \cdots & \psi_1(x^i_n)           &  0            & \cdots & 0             \\
      \vdots                  &        & \vdots                  &  \vdots       &        & \vdots        \\   
      \psi_m(x^i_1)           & \cdots & \psi_m(x^i_n)           &  0            & \cdots & 0             \\       
      \end{array}
\right]
\left[
  \begin{array}{c}
  w^i_1  \\
  \vdots \\
  w^i_n  \\
         \\ \hdashline
         \\
  \lambda_1 \\
  \vdots    \\
  \lambda_m \\
  \end{array}
\right]  
=
\left[
  \begin{array}{c}
  \mathcal{L}\left[ \phi(||x - x^i_1||) \right] \big|_{x=x_i} \\
  \vdots                                                      \\
  \mathcal{L}\left[ \phi(||x - x^i_n||) \right] \big|_{x=x_i} \\
                                                              \\ \hdashline
                                                              \\  
  \mathcal{L}\left[ \psi_1(x) \right] \big|_{x=x_i}           \\
  \vdots                                                      \\
  \mathcal{L}\left[ \psi_m(x) \right] \big|_{x=x_i}           \\
  \end{array}
\right]
\end{equation}
for each stencil. In eq. (\ref{eq:RBFFDWeights}), $\phi$ is a radial basis function (RBF) which we describe below, $||\bullet||$ indicates the $L_2$ norm, $\psi_i$ are monomial basis functions that span the space of all $d$-dimensional polynomials with a specified degree $p$ (e.g. $\{1, x, y\}$ for $d=2$ and $p=1$), and $\lambda_i$ are parameters that are estimated along with $w^i_j$ when eq. (\ref{eq:RBFFDWeights}) is inverted but they serve no purpose and can be discarded. 

Throughout this paper we use a cubic RBF for $\phi$,

\begin{equation}\label{eq:Cubic}
\phi(r) = r^3.
\end{equation}
The cubic RBF is an odd degree polyharmonic spline which has the benefit of being scale invariant and thus there is no scaling parameter that needs to be optimized, unlike for many other common choices of RBFs \citep[e.g.][]{Larsson2003}.  We note that the results presented in this paper remain virtually unchanged when we use other polyharmonic splines for $\phi$, which is consistent with the findings of \citet{Flyer2016}. 

We now elaborate on the stencil size, $n$, and the polynomial degree, $p$.  We choose $p$ to be equal to the degree of the derivative which we are approximating. This choice is based on the analysis of \citet{Flyer2016} and it ensures that eq. (\ref{eq:RBFFDApprox}) will converge to the true derivative as the distance between nodes decreases. The accuracy of eq. (\ref{eq:RBFFDApprox}) also generally improves with larger values of $n$, but at the expense of computational costs. We then choose $n$ to be large enough for eq. (\ref{eq:RBFFDApprox}) to converge. For the demonstrations in this paper, we find that $n=30$ is an appropriate choice.  It is worth noting that eq. (\ref{eq:RBFFDWeights}) cannot be inverted when the number of nodes is less than the number of monomial basis functions, $m$.  We then have a lower bound on $n$ which is $n \ge m = {{p+d}\choose{d}}$.   

As mentioned, the stencil $\mathbf{x^i}$ consists of the $n$ nearest neighbors to $x_i$.  An obvious assumption in eq. (\ref{eq:RBFFDApprox}) is that $u(x)$ is sufficiently smooth over the footprint of $\mathbf{x}^i$. If there is a known discontinuity in $u(x)$, then it is still possible for eq. (\ref{eq:RBFFDApprox}) to accurately approximate $\mathcal{L}[u(x)]$ as long as no stencils overlap the discontinuity. Programatically, this can be easily implemented by redefining the distance norm used to determine the nearest neighbors.  When there is a known discontinuity, we define the distance between two nodes to be the $L_2$ norm, unless the line segment connecting the two nodes intersects the discontinuity. If there is an intersection then the distance between the two nodes is considered infinite. We use this modified distance norm to account for the known creeping segment of the San Andreas fault in Section \ref{sec:ApplicationsSoCal}.  

\subsection{RBF-FD Filter}\label{sec:Filter}
We now introduce the RBF-FD filter which is a method for smoothing irregularly spaced $d$-dimensional data.  Existing methods for performing this task often fall in one of two categories.  One approach is to fit a set of basis functions to the observations using least-squares or regularized least-squares \citep[e.g.][]{Fasshauer2007}.  The second category is Gaussian process regression, which is a Bayesian technique where a stochastic prior model is assumed for the underlying signal \citep[e.g.][]{Rasmussen2006}.  Kriging is among the better known examples of Gaussian process regression \citep{Matheron1963}.  The RBF-FD filter is also an example of Gaussian process regression, and it distinguishes itself from other methods because it is computationally efficient and because it can be conventiently viewed as a low-pass filter with a user defined cut-off frequency. Furthermore, the RBF-FD filter can easily be designed so that the data is not smoothed across known discontinuities. In the following discussion of the RBF-FD filter, we seek to find a smoothed solution, $u_\mathrm{post}$, from the irregularly spaced, observed data, $u_\mathrm{obs}$. We constrain $u_\mathrm{post}$ with the observation equation

\begin{equation}\label{eq:Data}
  u_\mathrm{post} = u_\mathrm{obs} + \epsilon,\ \ \ \epsilon \sim \mathcal{N}(0,\mathbf{C}_\mathrm{obs}),
\end{equation}
and the prior model

\begin{equation}\label{eq:Prior}
  u_\mathrm{prior} \sim \mathcal{N}(0,\mathbf{C}_\mathrm{prior}),
\end{equation}
where $\epsilon$ and $u_\mathrm{prior}$ are considered to be Gaussian processes with zero mean and covariances $\mathbf{C}_\mathrm{obs}$ and $\mathbf{C}_\mathrm{prior}$ respectively.  The solution for $u_\mathrm{post}$ minimizes the objective function  

\begin{equation}\label{eq:Objective}
||u_\mathrm{post} - u_\mathrm{obs}||_{\mathbf{C}_\mathrm{obs}}^2 + 
||u_\mathrm{post}||_{\mathbf{C}_\mathrm{prior}}^2
\end{equation}
and is itself a Gaussian process with a distribution described by

\begin{equation}
  u_\mathrm{post} \sim \mathcal{N}(\bar{u}_\mathrm{post},\mathbf{C}_\mathrm{post}).
\end{equation}
We use $\bar{u}_\mathrm{post}$ and $\mathbf{C}_\mathrm{post}$ to denote the mean and covariance of $u_\mathrm{post}$ respectively.  Using Bayesian linear regression \citep{Tarantola2005} these values are found to be  

\begin{equation}\label{eq:GeneralSolution}
\begin{split}
  \bar{u}_\mathrm{post} &= (\mathbf{C}_\mathrm{obs}^{-1} + 
                            \mathbf{C}_\mathrm{prior}^{-1})^{-1}
                            \mathbf{C}_\mathrm{obs}^{-1} u_\mathrm{obs}
\\
\mathbf{C}_\mathrm{post} &= (\mathbf{C}_\mathrm{obs}^{-1} + 
                             \mathbf{C}_\mathrm{prior}^{-1})^{-1}.                          
\end{split}
\end{equation}
 
$\mathbf{C}_\mathrm{obs}$ is presumably well known, while $\mathbf{C}_\mathrm{prior}$ needs to be chosen based on an understanding of the underlying signal which we are trying to estimate. In Section \ref{sec:Smoothing1D} we discuss our choice for $\mathbf{C}_\mathrm{prior}$ and provide demonstrations with one-dimensional data.  The natural extension for $\mathbf{C}_\mathrm{prior}$ when dealing with $d$-dimensional data is discussed in Section \ref{sec:SmoothingND}.  

\subsubsection{Filtering in One Dimension}\label{sec:Smoothing1D}
For one-dimensional data we consider a prior which can be stated implicitly as

\begin{equation}\label{eq:ImplicitPrior1D}
  \mathbf{D}_{n} u_\mathrm{prior} = q, \ \ \ q \sim \mathcal{N}(0,\lambda^2),
\end{equation}  
where $\mathbf{D}_n$ is an $n$'th order differentiation matrix, and $q$ is white noise with constant variance $\lambda^2$.  If we momentarily ignore the fact that $\mathbf{D}_n$ is not invertible then we can explicitly write our prior covariance as

\begin{equation}\label{eq:ExplicitPrior1D}
\mathbf{C_\mathrm{prior}} = \lambda^2(\mathbf{D}_n^T\mathbf{D}_n)^{-1}.
\end{equation}
The filtered mean and covariance for the posterior are then 
\begin{equation}\label{eq:1DSolution}
\begin{split}
\bar{u}_\mathrm{post} &= (\mathbf{C}_\mathrm{obs}^{-1} +   
                   \frac{1}{\lambda^2}\mathbf{D}_n^T\mathbf{D}_n)^{-1}\mathbf{C}_\mathrm{obs}^{-1}
                   u_\mathrm{obs}
\\
\mathbf{C}_\mathrm{post} &= (\mathbf{C}_\mathrm{obs}^{-1} +   
                            \frac{1}{\lambda^2}\mathbf{D}_n^T\mathbf{D}_n)^{-1}.
\end{split}
\end{equation}
This filtered solution is closely tied to several well established methods of smoothing.  For example, one can immediately recognize eq. (\ref{eq:1DSolution}) as an example of Tikhonov regularization \citep{Tikhonov1978}. We also note a similarity between eq. (\ref{eq:1DSolution}) and smoothing splines \citep{Wahba1990}.  To see this similarity, we first recall that in one-dimension, a smoothing spline is defined as the function, $f(t)$, which minimizes,

\begin{equation}\label{eq:SmoothingSpline}
\sum_{i=1}^P (u_\mathrm{obs}^i - f(t_i))^2 + \alpha \int_{t_1}^{t_P} f^{(n)}(t) dt,
\end{equation}
where $u_\mathrm{obs}^i$ is an observation at time $t_i$, $P$ is the number of observations, $\alpha$ is a smoothing parameter, and $f^{(n)}$ denotes the $n$'th time derivative of $f$.
If we ignore data uncertainties (i.e. $\mathbf{C}_\mathrm{obs}=\mathbf{I}$), we can see from eq. (\ref{eq:Objective}) that $\bar{u}_\mathrm{post}$ is the discrete function which minimizes  

\begin{equation}\label{eq:1DObjective}
||u_\mathrm{obs} - \bar{u}_\mathrm{post}||_2^2 + \frac{1}{\lambda^2}||D_n\bar{u}_\mathrm{post}||_2^2.
\end{equation} 
If $u_\mathrm{obs}$ is sampled at evenly spaced increments then, with the appropriate choice of smoothing parameters, eq. (\ref{eq:1DObjective}) can be recognized as a discretized form of (\ref{eq:SmoothingSpline}) and we would thus expect $f(t)$ and $\bar{u}_\mathrm{post}$ to be effectively the same. The similarities break down when dealing with non-uniformly spaced observations.  

Whether we use a smoothing spline or eq. (\ref{eq:1DSolution}) to filter $u_\mathrm{obs}$, we must choose the appropriate penalty parameter.  One common method for choosing an appropriate penalty parameter is generalized cross-validation \citep{Craven1979}, which yields a filtered solution with the maximum predictive power.  There is merit to using an entirely objective approach such as cross-validation, and this would be approriate if there is no prior knowledge of the signal's characteristic wavelength.  Otherwise, it may be better to chose a penalty parameter that damps out all the high frequency oscillations which are known to be noise.  We elaborate on this point by demonstrating that eq. (\ref{eq:1DSolution}) can also be viewed as a low-pass filter with a cutoff frequency determined by $\lambda$.  

We wish to transform $\bar{u}_\mathrm{post}$ in eq. (\ref{eq:1DSolution}) to the frequency domain and in order to do so we assume that $u_\mathrm{obs}$ has a constant sampling rate. We also require that $\epsilon$ and $u_\mathrm{prior}$ are stationary stochastic processes (i.e. their statistical properties are invariant to time shifts).  We require them to be stationary so that matrix multiplication by $\mathbf{C}_\mathrm{obs}$ and $\mathbf{C}_\mathrm{prior}$ can be viewed as convolution in the time domain and thus multiplication in the frequency domain.  We then consider the simplifying case where $\epsilon$ is white noise with constant variance $\sigma^2$ (i.e. $\mathbf{C}_\mathrm{obs} = \sigma^2\mathbf{I}$), and $\mathbf{D}_n$ is the periodic spectral differentiation matrix \citep[e.g.][]{Trefethen2000}.  Under a discrete Fourier transform, $\mathbf{D}_n$ has the properties

\begin{equation}\label{eq:Property1}
  \mathcal{F}[\mathbf{D}_ng] = (2\pi i\omega)^n \hat{g}
\end{equation}
and

\begin{equation}\label{eq:Property2}
  \mathcal{F}[\mathbf{D}^T_ng] = (-2\pi i\omega)^n \hat{g},
\end{equation}
where $\omega$ is the frequency domain variable, $g$ is an arbitrary vector and $\hat{g}$ is its discrete Fourier transform.  The discrete Fourier transform of $\bar{u}_\mathrm{post}$ is then

\begin{equation}\label{eq:1DFourierSoln1}
\hat{u}_\mathrm{post}(\omega) = \frac{\frac{1}{\sigma^2}}
                                  {\frac{1}{\sigma^2} +                  
                                  \frac{(2\pi\omega)^{2n}}{\lambda^2}}
                                  \hat{u}_\mathrm{obs}(\omega).
\end{equation}
We make the change of variables 

\begin{equation}\label{eq:VariableChange}
\lambda^2 = (2\pi\omega_c)^{2n}\sigma^2
\end{equation}
which changes the hyperparameter from $\lambda$ to $\omega_c$.  The reason for this change of variables becomes apparent when we simplify eq. (\ref{eq:1DFourierSoln1}) to

\begin{equation}\label{eq:1DFourierSoln2}
\hat{u}_\mathrm{post}(\omega) = \frac{1}
                                  {1 + \left(\frac{\omega}{\omega_c}\right)^{2n}}
                                  \hat{u}_\mathrm{obs}(\omega).        
\end{equation}
We can recognize eq. (\ref{eq:1DFourierSoln2}) as an $n$'th order low-pass Butterworth filter with cut-off frequency $\omega_c$.  In the limit as $n\to \infty$ eq. (\ref{eq:1DFourierSoln2}) becomes an ideal low-pass filter which removes all frequencies above $\omega_c$ and leaves lower frequencies unaltered.  Of course, an ideal low-pass filter is often undesirable because it will tend to produce ringing artifacts in the filtered solutions.  When modeling $u_\mathrm{prior}$ as Brownian motion or integrated Brownian motion, where $n=1$ and $n=2$ respectively, the transfer function is tapered across $\omega_c$, which ameliorates ringing in the filtered solution.

By demonstrating that eq. (\ref{eq:1DSolution}) can be made equivalent to a low-pass filter, it may not be clear why we would ever use eq. (\ref{eq:1DSolution}) when it is far more efficient to filter in the frequency domain through the Fast Fourier Transform.  In order to make use of the Fast Fourier Transform, the observations must be made at a constant sampling rate and the observation noise must be  white with constant variance.  In contrast, these conditions do not need to be met in order to evaluate eq. (\ref{eq:1DSolution}).  The question is then whether eq. (\ref{eq:1DSolution}) still effectively acts as a low-pass filter when the idealized conditions are not met.  

We address this question here with two numerical demonstrations.  In this first demonstration we generate 100 samples of synthetic data over the time interval $0<t<1$ yr.  The true signal in the syntethic data is a sine wave with a frequency of 1 yr$^{-1}$ and 1 mm amplitude.  We obscure the synthetic data with white noise that does not have a constant variance.  We randomly pick a variance for each datum from a uniform random distribution ranging from 0.5 to 2 mm$^2$.  For the period from 1 to 2 years, we assign a variance of 25 mm$^2$ to the data, effectively making the data uniformitive for the filtered solution.  The synthetic data and its estimated power spectral density are plotted in Figure \ref{fig:Demo1}.  Our goal is to use eq. (\ref{eq:1DSolution}) as a low-pass filter which damps out frequencies that are higher than our signal.  Since our data does not have constant variance, we cannot choose a cutoff frequency from eq. (\ref{eq:VariableChange}).  Instead we define a characteristic data variance as

\begin{equation}
\frac{1}{\bar{\sigma^2}} = \frac{1}{P} \mathrm{tr}\left(\mathbf{C}_\mathrm{obs}^{-1}\right),
\end{equation}
where $P$ is the number of observations, and we relate $\lambda$ to $\omega_c$ by

\begin{equation}\label{eq:VariableChange2}
\lambda^2 = (2\pi\omega_c)^{2n}\bar{\sigma}^2.  
\end{equation}
In this demonstration $D_n$ is still the spectral differentiation matrix and we use $n=2$. Our cutoff frequency is chosen to be $\omega_c=2$ yr$^{-1}$.  The filtered solution and its frequency content are plotted Figure \ref{fig:Demo1}. Based on panel B, eq. (\ref{eq:1DSolution}) is indeed acting as a low-pass filter with the specified cut-off frequency even though the variances are not constant.  However, the frequency content of subsections of $\bar{u}_\mathrm{post}$ appears to change depending on the variance of $u_\mathrm{obs}$. For example, Over the interval $1<t<2$, where the variance of $u_\mathrm{obs}$ is higher, $u_\mathrm{post}$ lacks the higher frequency oscillations that can be seen in the remainder of the time series. This is desirable behavior because we do not want $u_\mathrm{post}$ contorting to fit dubious data.  It is also worth pointing out that $u_\mathrm{post}$ asymptotically approaches a solution that is not significantly different from that shown in Figure \ref{fig:Demo1} when the variance for $u_\mathrm{obs}$ increases to infinity over the interval $1<t<2$.  This too is desirable behaviour because one could then surmise that an appropriate way to handle missing data with eq. (\ref{eq:1DSolution}) is to assign $u_\mathrm{obs}$ an infinite variance for the missing time period.  With this insight, we can then extend the application of eq. (\ref{eq:1DSolution}) from data smoothing to data interpolation and extrapolation.   

\begin{figure}
\includegraphics[scale=0.75]{figures/fig1}
\caption{Panel A shows $u_\mathrm{obs}$ (black scatter points), $u_\mathrm{post}$ (blue line), and the true signal which we are trying to recover (red line).  The lines on each scatter point and the light blue region show the one standard deviation uncertainty for the observations and filtered solution respectively. Panel B shows the estimated power spectral density for the observed, filtered, and true signal.  The black dashed line is the squared transfer function in eq. (\ref{eq:1DFourierSoln2}) with $\omega_c=2$ yr$^{-1}$, which roughly indicates how eq. (\ref{eq:1DSolution}) scales the frequency content of $u_\mathrm{obs}$.  If the data variances were constant then the frequency content of $u_\mathrm{obs}$ and $\bar{u}_\mathrm{post}$ would be exactly related by the black dashed line.}   
\label{fig:Demo1}
\end{figure}

We present a similar demonstration to show that eq. (\ref{eq:1DSolution}) effectively acts as a low-pass filter even when the observations are not uniformly spaced.  We take samples over the time interval $(0,10)$ yr, where the sampling rate is $0.05$ yr$^{-1}$ for the first 5 years and $0.0125$ yr$^{-1}$ for the last 5 years.  We use the same underlying signal from the previous demonstration and we add noise with a constant variance of $1$ mm$^2$.  We keep $n=2$ but since we are dealing with non-uniformly spaced data, we can no longer use the periodic spectral differentiation matrix, and $\mathbf{D}_n$ is instead a first-order accurate finite difference matrix for arbitrarily spaced data, which can be computed with traditional finite difference methods \citet{Fornberg1996} or the procedure in Section \ref{sec:Differentiating}.  Once again, our cutoff frequency is $\omega_c = 2$ yr$^{-1}$.  The synthetic data and filtered solution are shown in panel A of Figure \ref{fig:Demo2}.  The power spectral density for the time intervals $(0,5)$ and $(5,10)$ yr are shown in panel B and C, respectively.  In both panels B and C, the spectral density of the filtered solution diverges from the spectral density of the synthetic observations at the expected cutoff frequency of $\omega_c$.  For frequencies above ${\sim}5$ yr$^{-1}$, the rate that the spectral content decays is slower than what is predicted by the transfer function in eq. (\ref{eq:1DFourierSoln2}). This is most pronounced in panel C, and we attribute the discrepancy to our lower order differentiation matrix. Barring the high frequencies which have negligable power in $\bar{u}_\mathrm{post}$, the power spectral density of $\bar{u}_\mathrm{post}$ relative to $u_\mathrm{obs}$ is accurately described by the transfer function in eq. (\ref{eq:1DFourierSoln2}), despite our use of non-uniform spacing.


\begin{figure}
\includegraphics[scale=0.7]{figures/fig2}
\caption{Panel A shows $u_\mathrm{obs}$ (black scatter points), $u_\mathrm{post}$ (blue line), and the true signal which we are trying to recover (red line).  The light blue region show the one standard deviation uncertainty for $u_\mathrm{post}$. The uncertainties for the observations are all 1 mm and are not shown for the sake of clarity.  Panel B shows the estimated power spectral density for the observed, filtered, and true signal over the time interval (0,5) yr. Panel C is the same as panel B except for the time interval (5,10) yr.  In both panel B and C the black dashed line is the squared transfer function in eq. (\ref{eq:1DFourierSoln2}) with $\omega_c=2$ yr$^{-1}$.}   
\label{fig:Demo2}
\end{figure}

\subsubsection{Filtering in Higher Dimensions}\label{sec:SmoothingND} 
We expand our discussion to smoothing data which is observed in $d$-dimensional space.  The filtered solution is still given by eq. (\ref{eq:GeneralSolution}), and we consider the prior model

\begin{equation}
  \mathbf{L}_n u_\mathrm{prior} = q, \ \ \ q \sim \mathcal{N}(0,\lambda^2)
\end{equation}  
where $\mathbf{L}_n$ is a differentiation matrix which approximates the operation 

\begin{equation}
  \sum_{i=1}^d\frac{\partial^n}{\partial x_i^n} 
\end{equation} 
and $n$ is an even integer. In general, we can construct $\mathbf{L}_n$ for arbitrarily spaced data with the strategy described in Section \ref{sec:Differentiating}. The corresponding covariance matrix is then

\begin{equation}\label{eq:NDCovariance}
\mathbf{C}_\mathrm{prior} = \lambda^2\left(\mathbf{L}_n^T\mathbf{L}_n\right)^{-1}. 
\end{equation}           
Using the change of variables from eq. (\ref{eq:VariableChange2}), the solution in the time domain is

\begin{equation}\label{eq:NDSolution}
\begin{split}
\bar{u}_\mathrm{post} &= (\mathbf{C}_\mathrm{obs}^{-1} +   
                   \frac{1}{(2\pi\omega_c)^{2n}\bar{\sigma}^2}\mathbf{L}_n^T\mathbf{L}_n)^{-1}\mathbf{C}_\mathrm{obs}^{-1}
                   u_\mathrm{obs}
\\
\mathbf{C}_\mathrm{post} &= (\mathbf{C}_\mathrm{obs}^{-1} +   
                            \frac{1}{(2\pi\omega_c)^{2n}\bar{\sigma}^2}\mathbf{L}_n^T\mathbf{L}_n)^{-1}.
\end{split}
\end{equation}
If we again assume that the observation are regularly spaced, have constant variance, and $\mathbf{L}_n$ is the corresponding spectral differentiation matrix, then the d-dimensional discrete Fourier transform of $\bar{u}_\mathrm{post}$ is 

\begin{equation}\label{eq:NDFourierSoln}
  \hat{u}_\mathrm{post}(\omega_1, \dots, \omega_d) = 
  \frac{1}{1 + \left(\sum_{i=1}^d \left(\frac{\omega_i}{\omega_c}\right)^n\right)^2} \hat{u}_\mathrm{obs}.
\end{equation}
The transfer function in eq. (\ref{eq:NDFourierSoln}) can once again be recognized as a low-pass filter.  Namely, in the limit as $n \to \infty$, the transfer function becomes a $d$-dimensional box which is zero for all the frequency tuples, $(\omega_1,\dots,\omega_n)$, which have a component greater than $\omega_c$.

When the data is not regularly spaced, eq. (\ref{eq:NDSolution}) is no longer a low-pass filter, strictly speaking, because the Discrete Fourier Transform is not clearly defined in that case. Nonetheless, numerical tests verify that the shortest feature wavelength retained in the filtered solution is consistent with the specified cutoff frequency, regardless of the data spacing.  For this reason, we argue that eq. (\ref{eq:NDSolution}) effectively acts as a low-pass filter for irregularly spaced, $d$-dimensional data.

\section{Applications}\label{sec:Applications}

\subsection{Strain Rate in Southern California}\label{sec:ApplicationsSoCal}
give definition of strain
remark on how uncertainties are computed and also how they dont account for error in L

\subsection{Time Depenent Strain Rate in Cascadia}\label{sec:ApplicationsCascadia}

\section{Discussion and Conclusion}\label{sec:Discussion}
% discuss potential applications for slip models

% discuss how we remove errors like reference frame and seasonals

\bibliographystyle{apalike}
\bibliography{mybib}  
 
\end{document}