\documentclass[10pt,a4paper]{article}
\usepackage[utf8]{inputenc}
\usepackage{amsmath}
\usepackage{amsfonts}
\usepackage{amssymb}
\usepackage[left=2cm,right=2cm,top=2cm,bottom=2cm]{geometry}
\begin{document}

In the most general sense, we seek to estimate a smoothed signal, $u_\mathrm{smooth}$, from observed data, $u_\mathrm{obs}$, by incorporating prior knowledge of the signals statistical properties.  In this paper, we describe the smoothed and observed signal as 
\begin{equation}\label{eq:Data}
  u_\mathrm{smooth} = u_\mathrm{obs} + \epsilon,\ \ \ \epsilon \sim \mathcal{N}(0,\mathbf{C}_\mathrm{obs})
\end{equation}
\begin{equation}\label{eq:Prior}
  u_\mathrm{smooth} = u_\mathrm{prior}, \ \ \ u_\mathrm{prior} \sim \mathcal{N}(0,\mathbf{C}_\mathrm{prior}).
\end{equation}

We find $u_\mathrm{smooth}$ that simulataneously satisfies eq. (\ref{eq:Data}) and eq. (\ref{eq:Prior}) in a least-squares sense by minimization the objection function
\begin{equation}\label{eq:Objective}
F(u_\mathrm{smooth}) = ||u_\mathrm{smooth} - u_\mathrm{obs}||_{\mathbf{C}_\mathrm{obs}}^2 + 
                       ||u_\mathrm{smooth}||_{\mathbf{C}_\mathrm{prior}}^2 .
\end{equation} 
The solution that minimizes eq. \ref{eq:Objective} is simply
\begin{equation}
u_\mathrm{smooth} = (\mathbf{C}_\mathrm{obs}^{-1} + 
                     \mathbf{C}_\mathrm{prior}^{-1})^{-1}
                     \mathbf{C}_\mathrm{obs}^{-1} u_\mathrm{obs} .
\end{equation} 
The challenge is in choosing an appropriate $\mathbf{C}_\mathrm{prior}$.  Below we discuss selection of $\mathbf{C}_\mathrm{prior}$ is one-dimensional problems, which then naturally leads to an extension to selection of $\mathbf{C}_\mathrm{prior}$ for higher dimensions.

\section*{One-dimensional smoothing}
When the signal is only assumed to covary in time, $u_\mathrm{prior}$ is commonly treated as Brownian motion (eg. Me Segall, McQuire, Murray, etc.).  We treat a $u_\mathrm{prior}$ as Brownian motion by assuming that its velocity is white noise with constant variance $\lambda^2$. That is to say
\begin{equation}
  \mathbf{D}_1 u_\mathrm{prior} = q, \ \ \ q \sim \mathcal{N}(0,\lambda^2)
\end{equation}     
where $\mathbf{D}_N$ denotes an $N$'th order differentiation matrix.  We then model $u_\mathrm{prior}$ as Brownian motion by setting $\mathbf{C}_\mathrm{prior}$ to be
\begin{equation}
\mathbf{C_\mathrm{prior}} = \lambda^2(\mathbf{D}_1^T\mathbf{D}_1)^{-1}.
\end{equation}
It is also common to treate $u_\mathrm{prior}$ as integrated Brownian motion. In such case, the appropriate choice of $\mathbf{C}_\mathrm{prior}$ would just use $\mathbf{D}_2$ rather than $\mathbf{D}_1$.  There is still a need to select $\lambda$, which described how rapidly we expect the smoothed signal to vary.  There are numerous methods for selecting $\lambda$.  For example, one could use maximum likelihood methods, a trade-off curve, or simply vary $\sigma$ until the smoothed signal looks appropriate when compared to the observations.  We note that smoothing is fundamentally a low-pass filter and so it is perhaps most useful to choose $\sigma$ based on its cut-off attenuation frequencies.  We then consider the solution for $u_\mathrm{smooth}$ in the frequency domain.  

For the purpose analytical tractability, we assume that $u_\mathrm{smooth}$ and $\epsilon$ are stationary stochastic processes and furthermore
\begin{equation}
\mathbf{C}_\mathrm{obs} = \sigma^2 \mathbf{I}
\end{equation}  
and
\begin{equation}
\mathbf{C}_\mathrm{prior} = \lambda^2 (\mathbf{D}_N^T\mathbf{D}_N)^{-1}.
\end{equation}
The solution for $u_\mathrm{smooth}$ in the frequency domain is
\begin{equation}\label{eq:Fourier1}
\hat{u}_\mathrm{smooth}(\omega) = \frac{\left(\frac{1}{\sigma}\right)^2}
                                  {\left(\frac{1}{\sigma}\right)^2 + \left(\frac{(2\pi\omega)^N}{\lambda}\right)^2}
                                  \hat{u}_\mathrm{obs}(\omega).
\end{equation}
We make the change of variables
\begin{equation}\label{eq:VariableChange}
\lambda = (2\pi\omega_c)^N\sigma
\end{equation}
where $\omega_c$ is now our free parameter. This simplifies eq. \ref{eq:Fourier1} to
\begin{equation}\label{eq:Fourier2}
\hat{u}_\mathrm{smooth}(\omega) = \frac{1}
                                  {1 + \left(\frac{\omega}{\omega_c}\right)^{2N}}
                                  \hat{u}_\mathrm{obs}(\omega).
\end{equation}
We can recognize eq. \ref{eq:Fourier2} as a N'th order Butterworth filter, and $\omega_c$ is the cut-off frequency.  Thus selecting the hyperparameter, $\lambda$, can be translated through eq. \ref{eq:VariableChange} into a more tangible question of identifying an upper bound on the frequency content of the underlying signal. In the limit as $N\to \infty$ eq. \ref{eq:Fourier2} becomes an ideal low-pass filter which removes all frequencies above $\omega_c$ and leaves lower frequencies unaltered.  Of course, an ideal low-pass filter is often undesirable in practice because it will tend to produce ringing artifacts in the smoothed solutions.  When modeling $u_\mathrm{prior}$ as a Brownian motion or integrated Brownian motion, where $N=1$ and $N=2$ respectively, the transfer function is tapered across $\omega_c$, which ameliorates ringing in the smoothed solution.  

\end{document}